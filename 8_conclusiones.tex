% !TeX root = Tesis.tex
% !TeX encoding = UTF-8
% !TeX spellcheck = es
\chapter*{Conclusiones}\label{conclutions}
\addcontentsline{toc}{chapter}{Conclusiones}
\onehalfspacing

Nuestros sencillos esquemas cualitativos identifican direcciones en el espacio de expresión genética asociadas a diferentes procesos biológicos: envejecimiento, carcinogénesis, aparición de la enfermedad de Alzheimer. Cada una de estas direcciones se caracteriza por un "metagén" o perfil de expresión genética, del cual se pueden extraer los principales genes que contribuyen al proceso.

Algunos de nuestros resultados confirman conocimientos previos, pero otros requieren mayor corroboración. Por ejemplo, la idea de que el envejecimiento programático podría estar relacionado con evitar el fuerte atractor GB, o la aparición tardía de la EA como la captura por el atractor de EA de muestras de edad normal. Esperamos que estos resultados motiven la investigación experimental en estas direcciones. El experimento en un modelo de ratones es particularmente factible, como lo demuestra la referencia \cite{hahn2023atlas}.

Cabe destacar que incluso datos o métodos computacionales más refinados no podrían modificar sustancialmente nuestros esquemas cualitativos con solo tres atractores. Sus posiciones relativas podrían variar, pero las afirmaciones formuladas se mantendrán.


\chapter*{Recomendaciones}\label{recomendations}
\addcontentsline{toc}{chapter}{Recomendaciones}

%TODO: Aquí se puede incluir la propuesta de comprobar los resultados de forma experimental, haciendo alución a que en un modelo de ratón no debe ser muy complicado.

%TODO: También se podría hacer una simulación para comprobar que debe haber una presión selectiva que marque la ruta que debe seguir un sistema en el camino hacia el envejecimiento, evitando el fuerte atractor del GB

El diagrama de la Fig. \ref{fig:fig1a} debe completarse con datos correspondientes a otros tipos de demencia o trastornos cerebrales. En particular, se espera un área de la enfermedad de Parkinson cercana al atractor de la EA y opuesta al GB \cite{Mencke_2020}. El panorama completo puede revelar una topología aún más precisa del espacio de expresión genética y un diagrama de Wright más completo.

Anticipamos que diagramas similares en otros tejidos, además de proporcionar una perspectiva integral, podrían ser útiles en la comprensión de la biología de enfermedades o trastornos aparentemente no relacionados, y en el descubrimiento de pistas inesperadas para su tratamiento.
