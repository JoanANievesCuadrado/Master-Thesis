% !TeX root = Tesis.tex
% !TeX spellcheck = es
% !TeX encoding = UTF-8

\chapter{}\label{apx:apx6}

\begin{table}[!htb]
	\centering
	\caption{Los principales procesos de \textit{Reactome} relacionados con los 100 primeros genes en el ranking a lo largo de la dirección PC2}
	\label{tab:apx6}
	\begin{tabular}{|c|c|c|c|c|c|c|}
		\hline
		\multirow{2}{*}{Proceso} & \multicolumn{4}{c|}{Entidades} & \multicolumn{2}{c|}{Reacciones} \\ \cline{2-7}
		& encontrado & tasa & \textit{p-value} & FDR & encontrado & tasa \\ \hline
		
		G0 y G1 temprano & $4/38$ & $0.002$ & $1.58 \cdot 10 ^{-04}$ & $0.032$ & $4/27$ & $0.002$ \\ \hline
		
		TFAP2A actúa como un represor & \multirow{4}{*}{$20/344$} & \multirow{4}{*}{$0.022$} & \multirow{4}{*}{$3.72 \cdot 10^{-13}$} & \multirow{4}{*}{$9.16 \cdot 10^{-11}$} & \multirow{4}{*}{$63/167$} & \multirow{4}{*}{$0.012$} \\
		transcripcional durante & & & & & & \\
		la diferenciación celular & & & & & & \\
		inducida por ácido retinoico  & & & & & & \\\hline
		
		
		
		
	\end{tabular}
\end{table}


