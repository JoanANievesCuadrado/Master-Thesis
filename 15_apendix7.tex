% !TeX root = Tesis.tex
% !TeX spellcheck = es
% !TeX encoding = UTF-8

\chapter{}\label{apx:apx7}

\begin{table}[!htb]
	\centering
	\caption{Los 10 primeros genes en el \textit{ranking} de PCA para la transición de O a la EA.}
	\label{tab:apx7}
	\begin{tabular}{|c|c|c|}
		\hline
		Gen & Nombre & PC1 \\ \hline
		
		\hline
		SNAP25 & \textit{Synaptosome Associated Protein 25} & 1.00 \\
		\hline
		VSNL1 & \textit{Visinin Like 1} & 0.90 \\
		\hline
		STMN2 & \textit{Stathmin 2} & 0.90 \\
		\hline
		ENC1 & \textit{Ectodermal-Neural Cortex 1} & 0.89 \\
		\hline
		NEFL & \textit{Neurofilament Light Chain} & 0.89 \\
		\hline
		SYT1 & \textit{Synaptotagmin 1} & 0.88 \\
		\hline
		RGS4 & \textit{Regulator Of G Protein Signaling 4} & 0.86 \\
		\hline
		CHN1 & \textit{Chimerin 1} & 0.84 \\
		\hline
		GABRA1 & \textit{Gamma-Aminobutyric Acid Type A Receptor Subunit Alpha1} & 0.77 \\
		\hline
		GABRG2 & \textit{Gamma-Aminobutyric Acid Type A Receptor Subunit Gamma2} & 0.77 \\
		\hline
		
	\end{tabular}
\end{table}

Los resultados se basan en cálculos de la referencia \cite{Gonzalez_2021}. Los pesos en el vector unitario a lo largo de PC1 se normalizan al valor más alto.
