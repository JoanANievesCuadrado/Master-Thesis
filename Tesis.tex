% !TeX document-id = {041f4270-ff29-4194-ac21-3def5a3e2038}
% !TeX encoding = UTF-8
% !TeX spellcheck = es_ES
% !TeX root = Tesis.tex
\documentclass[12pt,oneside,openleft]{book}

\usepackage{soul}
\usepackage{easyReview}

\usepackage[letterpaper,lmargin=2.9cm,rmargin=1.9cm,bmargin=2.8cm,tmargin=2.8cm,includeheadfoot]{geometry} % adjusts page layout, showframe

\usepackage{graphicx}  % Add graphics capabilities
\graphicspath{/figures/}
\usepackage{tikz, tikz-3dplot}
\usepackage{flafter}  % Don't place floats before their definition
\usepackage[list=off,font=small,skip=3pt]{caption}
\usepackage{subcaption}

\usepackage{lipsum}
\usepackage{float}

%\usepackage{natbib} % use author/date bibliographic citations
\usepackage{cite}

\usepackage{amsthm}
\usepackage{amsfonts}
\usepackage{amsmath,amssymb}  % Better maths support & more symbols
\usepackage{textcomp} % provide lots of new symbols
\usepackage{bm}  % Define \bm{} to use bold math fonts

\usepackage[utf8]{inputenc} % Any characters can be typed directly from the keyboard
%\usepackage[T1]{fontenc}
\usepackage[spanish]{babel}
\usepackage{color}
\usepackage{xcolor}
\usepackage{colortbl}
\usepackage{longtable}

\usepackage{booktabs}
\setlength{\tabcolsep}{4pt} % General space between cols (6pt standard)
\renewcommand{\arraystretch}{1} % General space between rows (1 standard)

\usepackage{enumitem}
\setlist{itemsep=0.5pt}%nolistsep


\usepackage[plain]{fancyref}
\usepackage[bookmarksopen,colorlinks,pdftex]{hyperref}  % PDF hyperlinks
\hypersetup{citecolor=black,
	        filecolor=black,
	        linkcolor=black,
	        urlcolor=black} % black links, for printed output

\usepackage{fancyhdr}
\fancyhead{\headheight = 14pt}
\fancyhead[LE,RO]{} %\slshape \nouppercase{\rightmark}}
\fancyhead[LO,RE]{\slshape \nouppercase{\leftmark}}
\fancyfoot[C]{\thepage}

\usepackage{aas_macros} %para bibliografía \apj, etc
\usepackage[]{tocbibind} % [nottoc] para remover el indice del toc

%\renewcommand{\baselinestretch}{1.5} % para cambiar interlineado
\usepackage{setspace}
%\usepackage{ieeetran}

\setcounter{secnumdepth}{4} % para numerar subsubsections y parrafos y agregar al toc
\setcounter{tocdepth}{4}
\newcommand{\myparagraph}[1]{\paragraph{#1}\mbox{}\\}

\usepackage{multirow}
\newcommand{\sfref}[1]{{\bfseries figura suplementaria \ref{#1}}}
%\newcommand{\Sfref}[1]{Supplementary figure \ref{#1}}
\newcommand{\stref}[1]{{\bfseries tabla suplementaria \ref{#1}}}
\newcommand{\stsref}[2]{{\bfseries tablas suplementarias \ref{#1} y \ref{#2}}}
\definecolor{tcA}{rgb}{0.811765,0.811765,0.811765}

\makeatletter %eqs en negrita si el texto q lo rodea lo está
\DeclareRobustCommand*{\bfseries}{%
  \not@math@alphabet\bfseries\mathbf
  \fontseries\bfdefault\selectfont
  \boldmath
}
\makeatother

% Para escribir elementos químicos
%\usepackage{chemmacros}
%\usepackage{bohr}
%\newcommand*\mychemistry[2]{%
%  \ch{^{#2}_{\atomicnumber{#1}}#1}}
%\newcommand*\mychemistryg[3]{%
%  \ch{^{#2}_{#3} #1}}
%
%\hyphenation{aproxi-ma-da-men-te}
%\hyphenation{Chandrasekhar}
%\hyphenation{Schwarzschild}

\begin{document}
	
	\selectlanguage{spanish}

    \renewcommand{\tablename}{Tabla}
     \renewcommand{\refname}{Referencias}
    \renewcommand{\figurename}{Figura}

    \renewcommand{\thefootnote}{\Roman{footnote}}

    \lefthyphenmin=2
    \righthyphenmin=2

    \frontmatter

    % !TeX root = Tesis.tex
% !TeX spellcheck = es_ES
% !TeX encoding = UTF-8
\pagestyle{empty}

\newcommand{\HRule}{\rule{\linewidth}{1pt}} % Defines a new command for the horizontal lines, change thickness here

\begin{titlepage}
	
{\begingroup % Create the command for including the title page in the document
	\hbox{ % Horizontal box
%		\hspace*{0.01\textwidth} % Whitespace to the left of the title page
		\rule{1pt}{\textheight} % Vertical line
		\hspace*{0.025\textwidth} % Whitespace between the vertical line and title page text
		\parbox[b][\textheight]{\textwidth}{ % Paragraph box which restricts text to less than the width of the page	
				
\begin{center}			


%\vfill

\Large

Instituto de Cibernética, Matemática y Física 

\smallskip

Departamento de Física Teórica

\vspace{8mm}

%\centerline{\mbox{
%\includegraphics[height=35mm,keepaspectratio]{./logos/logosFF}}}
%\begin{figure}[!h]
%	\begin{subfigure}
%		\includegraphics[height=35mm,keepaspectratio]{./logos/logosFF}
%	\end{subfigure}
%	\begin{subfigure}
%		\includegraphics[height=35mm,keepaspectratio]{./logos/logosUH}
%	\end{subfigure}
%\end{figure}


\begin{minipage}{0pt}
\end{minipage}
\hfill
\begin{minipage}[l]{.18\textwidth}
	\centering
%	\includegraphics[height=\linewidth,width=\linewidth]{./logos/logoFF}
\end{minipage}
\begin{minipage}[center]{.5\linewidth}
	\centering
	{\Large TESIS DE MAESTRÍA%\\[5pt] \emph{presentada en opción al grado científico de}  \\[5pt] \textbf{Licenciado en Física} % 
	}
\end{minipage}
\begin{minipage}[r]{.18\textwidth}
	\centering
%	\includegraphics[height=\linewidth,keepaspectratio]{./logos/logoUH}
\end{minipage}\hspace*{10pt}
\hfill
\begin{minipage}{0pt}
\end{minipage}
%\hspace{4pt}


\vfill

% \\[5pt]

\HRule \medskip

%{\LARGE \bfseries \MakeUppercase{Una vista panorámica a los atractores homeostático, Alzheimer y Glioblastoma} %[5pt] %\MakeUppercase{}
%}

{\LARGE \bfseries \MakeUppercase{Envejecimiento, Alzheimer y Glioblastoma en el espacio de expresión genética} %[5pt] %\MakeUppercase{}
}


\smallskip
\HRule

%%%%%%% FIGURE%%%%%%%%

%%%%%%%%%%%%%%%%%%%%

\vfill

\begin{tabular}{rl}
	
\medskip
\textbf{Autor:} & Joan Andrés Nieves Cuadrado\\
\noalign{\vspace{5pt}}

\textbf{Tutor:}& Dr. Augusto González García,  \textit{ICIMAF}\\ 
			      
\noalign{\vspace{2mm}}
\end{tabular}

\vfill
\begin{minipage}{\linewidth}
	\centering
	\includegraphics[height=.17\linewidth, keepaspectratio]{./logos/logoICIMAF.png}
\end{minipage}\\[5pt]
\vspace{10pt}
\vfill

\Large
La Habana, 2025


\end{center}

		}}
		\endgroup}

\end{titlepage}
\cleardoublepage

    % !TeX root = Tesis.tex
% !TeX encoding = UTF-8
% !TeX spellcheck = es_ES
\indent
%\section*{Agradecimientos}
%\addcontentsline{toc}{chapter}{Agradecimientos}

\vfill

\begin{quote}
\begin{flushright}
%	\textit{A veces, aceptar ayuda cuesta m\'as que ofrecerla.}\\
%	\textit{Luminara Undulis}
	\textit{Aquí va la dedicatoria o una frase cool}
\end{flushright}



\end{quote}

\vfill
\cleardoublepage

\pagestyle{plain}

	%%\begin{center}
%	\section*{Agradecimientos}
%	%\addcontentsline{toc}{chapter}{Agradecimientos}
%\end{center}
%
%\vfill
%
%\begin{quote}
%	
%Gracias a mi familia, amigos y tutores.
%
%
%
%\end{quote}
%
%\vfill
%\cleardoublepage
%
%\pagestyle{plain}

	% !TeX root = Tesis.tex
% !TeX encoding = UTF-8
% !TeX spellcheck = es_ES
\vfill

\begin{center}
\section*{Resumen}
\addcontentsline{toc}{chapter}{Resumen}
\end{center}
\medskip

\begin{minipage}[c]{.9\linewidth}
\begin{quote}
	
{\large Los datos disponibles de la materia blanca de cerebro permiten localizar los atractores normal (homeostático), Glioblastoma y Alzheimer en el espacio de expresión genética e identificar caminos relacionados con transiciones como la carcinogénesis o la aparición del Alzheimer. También se aprecia una trayectoria predefinida para el envejecimiento, lo cual es consistente con la hipótesis del envejecimiento programado. Adicionalmente, suposiciones razonables sobre la fortaleza relativa de los atractores permite dibujar un panorama esquemático del \textit{fitnest}: diagrama de Wright. Estos sencillos diagramas reproducen relaciones conocidas entre el envejecimiento, el Glioblastoma y el Alzheimer, y plantea cuestiones interesantes como la posible conexión entre el envejecimiento programado y el Glioblastoma en este tejido. Prevemos que múltiples diagramas similares en otros tejidos podrían ser útiles en el entendimiento de la biología de enfermedades o trastornos aparentemente no relacionados, y para descubrir pistas inesperadas para su tratamiento.}




\end{quote}
\end{minipage}

\smallskip
\vfill
\cleardoublepage

\begin{center}
\section*{Abstract}
\addcontentsline{toc}{chapter}{Abstract}
\end{center}
\medskip

\begin{quote}

{\large Available data for white matter of the brain allows to locate the normal (homeostatic), Glioblastoma and Alzheimer’s disease attractors in gene expression space and to identify paths related to transitions like carcinogenesis or Alzheimer’s disease onset. A predefined path for aging is also apparent, which is consistent with the hypothesis of programmatic aging. In addition, reasonable assumptions about the relative strengths of attractors allow to draw a schematic landscape of fitness: a Wright’s diagram. These simple diagrams reproduce known relations between aging, Glioblastoma and Alzheimer’s disease, and rise interesting questions like the possible connection between programmatic aging and Glioblastoma in this tissue. We anticipate that similar multiple diagrams in other tissues could be useful in the understanding of the biology of apparently unrelated diseases or disorders, and in the discovery of unexpected clues for their treatment.
}


\end{quote}

\vfill
\cleardoublepage

	
    \pagestyle{empty}%fancy

    \singlespacing
    \tableofcontents
	\onehalfspacing
    %\listoffigures
    %\listoftables

    \mainmatter
    \pagestyle{fancy}
    \pagestyle{plain}
    % !TeX root = Tesis.tex
% !TeX encoding = UTF-8
% !TeX spellcheck = es_ES
\chapter*{Introducción} \label{intro}
\addcontentsline{toc}{chapter}{Introducción}
\onehalfspacing

\lipsum[1-3]
    % !TeX root = Tesis.tex
% !TeX spellcheck = es_ES
% !TeX encoding = UTF-8
\chapter{Materiales y métodos}
\label{cap1}
\section{Análisis de Componentes Principales}
\label{c11}
\onehalfspacing

Una de las primeras estrategias utilizadas en el análisis de grandes bases de datos es la reducción de la dimensionalidad. Esto permite obtener representaciones en solo 2 o 3 dimensiones y de esta forma comenzar a ganar claridad sobre el estudio que se desea realizar. Algunas de las técnicas comunes de reducción dimensional son PCA (\textit{Principal Component Analisis}), t-SNE (\textit{t-Distributed Stochastic Neighbor Embedding}), UMAP (\textit{Uniform Manifold Approximation and Projection}), entre otras.

t-SNE es un método no linear y estocástico. Su funcionamiento se puede separar en dos etapas. En la primera, se seleccionan los vecinos de cada punto. Esto se hace utilizando una distribución gaussiana alrededor de cada punto, donde los más cercanos tienen un probabilidad mayor de ser seleccionados que los lejanos. Durante la segunda etapa se asignan posiciones iniciales aleatorias en un espacio de menor dimensión (generalmente 2 o 3 dimensiones). Luego, se define una distribución de probabilidad similar para los puntos en el nuevo espacio y se minimiza la divergencia entre las dos distribuciones. De esta forma el algoritmo logra obtiene una representación en un dimensión reducida que preserva la similaridad entre los vecinos cercanos.

UMAP es una técnica muy similar a t-SNE. Una de las diferencias principales es que, durante la selección de los vecinos se asume que los datos forman una variedad de menor dimensión que los espacio original. Otra característica de este método es que, para hacer la representación reducida minimiza el entropia cruzada en lugar de la divergencia entre las distribuciones. Estas diferencias le permiten a este algoritmo preservar mejor tanto la estructura local como la global de los datos. 

    % !TeX root = Tesis.tex
% !TeX encoding = UTF-8
% !TeX spellcheck = es
\chapter{Diagrama de tres atractores}
\label{cap2}
\onehalfspacing


\section{El diagrama de N + GB + EA}\label{sec:ngbad}

Nuestro punto de partida es el diagrama de los datos de expresión genética del análisis de componentes principales para la materia blanca del cerebro, mostrado en la Fig. \ref{fig:fig1a}. Como se puede notar en la figura, las dos primeras componentes principales capturan más del 80 \% de la varianza del sistema. Por lo tanto, es una representación bidimensional adecuada de la distribución real de los puntos en el espacio de expresión genética.

\begin{figure}[!htb]
	\centering
	\includegraphics[width=0.75\linewidth]{figures/Fig_1a.pdf}
	\caption{\label{fig:fig1a}
		Análisis de componentes principales para los datos de expresión genética.}
\end{figure}

En la figura se pueden apreciar 4 grupos de muestras. Las muestras marcadas como N y GB corresponden, a especímenes patológicamente normales y tumorales en los datos del TCGA para el glioblastoma \cite{Brennan_2013}. Los centros de las nubes de muestras de N y GB en el espacio de expresión genética definen, respectivamente, los atractores Normal (homeostático) y Glioblastoma de Kauffman \cite{Huang_2009, Gonzalez_2023}. De hecho, la acumulación de puntos en una determinada región de este espacio indica que esta es un atractor de la red de regulación Genética que gobierna la dinámica del sistema.

Por otro lado, los grupos etiquetados como EA y O corresponden a las muestras de la materia blanca del cerebro de la enfermedad del Alzheimer y del grupo de control (\textit{old}) en el estudio del Instituto Allen \cite{Miller_2017}. 

La Fig. \ref{fig:pcaotoad} es una reconstrucción de la figura 3 de la referencia \cite{Gonzalez_2021}. En esta se muestra los resultados del PCA para los datos de expresión genética de la materia blanca del cerebro del Instituto Allen. La primera componente principal (PC1), la cual contiene el $24.7 \%$ de la varianza total, discrimina entre las muestra de O y EA. La posición del centro de la nube de O en este eje es $\left\langle x_1 \right\rangle  = 0 $, y para la EA es $\left\langle x_1 \right\rangle = 40.97 $. Sin embargo, los radios de las nubes de las muestras de O y EA son más grandes que la distancia entre los centros, que son $80.69$ y $72.64$ respectivamente.
%\alert{Estudiamos la transición de O hacia EA en la referencia \cite{Gonzalez_2021}}. En la Fig. \ref{fig:otoad}

\begin{figure}[!htb]
	%TODO:
	\centering
	\includegraphics[width=0.75\linewidth]{figures/pca_o_to_ad_1}
	\caption{ Análisis de componentes principales de los datos de expresión genética del Instituto Allen de la materia blanca del cerebro. Se representan las muestras de O y la EA. Las elipses discontinuas se dibujan de acuerdo con las desviaciones estándar de cada conjunto. Tomada de la referencia \cite{Gonzalez_2021}.}
	\label{fig:pcaotoad}
\end{figure}

Es bien conocido el papel de la edad en la EA, especialmente en ancianos \cite{alz2019}. Por lo tanto, podemos usar la edad como una variable de tiempo para seguir la transición. A pesar del número relativamente pequeño de muestras, se realizó un análisis de regresión lineal de la posición media de $\left\langle x_1 \right\rangle$ en función de la edad en las muestras de O, Fig. \ref{fig:otoad}, muestra que $\left\langle x_1 \right\rangle = -287.12 + 3.24 \cdot edad$. En las muestras de la EA, sin embargo, no se encontró correlación entre $\left\langle x_1 \right\rangle$ y la edad observada. Por lo tanto, la posición de la zona EA es aproximadamente fija, y la nube de muestras de O muestra una deriva hacia el mínimo de EA a medida que aumenta la edad.

\begin{figure}[!htb]
	%TODO
	\centering
	\includegraphics[width=0.75\linewidth]{figures/O_to_AD_1}
	\caption{Posición media de la muestra a lo largo del eje PC1 en función de la edad. A medida que aumenta la edad, las muestras de O experimentan una deriva hacia la región de la EA, cuyo centro es aproximadamente independiente de la edad. Tomada de la referencia \cite{Gonzalez_2021}.}
	\label{fig:otoad}
\end{figure}

Una mejor ilustración de este hecho viene representada en la Fig. \ref{fig:supplotoad}, donde se compara la densidad de probabilidad de las muestras de O y de la EA. Se definen cuatro intervalos de edades, que contienen aproximadamente la misma cantidad de muestras de O: [77, 84], [84, 90], [90, 95], [95, 100+]. La probabilidad total de las muestras de la EA es mostrada en los cuatro paneles. Es aparente un desplazamiento de las muestras de O hacia la zona de la EA.


\begin{figure}[!htb]
	%TODO: hacer un replot de la figura
	\centering
	\includegraphics[width=0.75\linewidth]{figures/suppl_otoad}
	\caption{Densidad de probabilidad de las muestras de O y de la EA a lo largo del eje de PC1. Cada panel es para un intervalo de edad para las muestras de O. La probabilidad de la EA, la cual es aproximadamente independiente de la edad, es mostrada en los cuatro paneles. Tomada de la referencia \cite{Gonzalez_2021}.}
	\label{fig:supplotoad}
\end{figure}

Esta propiedad sugiere que el centro de la nube de muestras de la EA define un atractor en el espacio de expresión genética. Las muestras de O parecen ser atrapadas por el atractor de la EA en el proceso del envejecimiento.

Así, en nuestra aproximación, obtenemos un panorama en el espacio de expresión genética de tres atractores: N, GB y EA, y un conjunto de muestras de O que se desplaza hacia la EA. Las posiciones relativas y las principales transiciones entre los atractores se resumen en la Fig. \ref{fig:fig1b}. Asumimos que estas transiciones están determinadas por la biología subyacente a los procesos en los tejidos. La transición de N a la EA se denomina ``EA anticipada'' para enfatizar que también existe una vía hacia la EA a través del envejecimiento: ``EA tardía''. La figura también indica una vía para el GB y para el envejecimiento.

\begin{figure}[!htb]
	\centering
	\includegraphics[width=0.75\linewidth]{figures/Fig_1b.pdf}
	\caption{Posiciones relativas y principales transiciones entre los atractores.}
	\label{fig:fig1b}
\end{figure}

\section{Panorama del \textit{fitness}}\label{sec:sec22}

Existe información cualitativa que puede introducirse en nuestra descripción. Esta se relaciona con una variable de \emph{fitness}, de modo que dibujamos una especie de diagrama de Wright \cite{wright1932roles}. En la Fig. \ref{fig:fig1c} se muestra un diagrama esquemático que contiene un gráfico de contorno hipotético del \emph{fitness}. Los atractores N y GB son máximos de \emph{fitness} y deberían estar separados por una barrera de bajo \emph{fitness} \cite{Gonzalez_2021}. El GB debería ser el máximo más alto de los tres actores representados \cite{Gonzalez_2021, gonzalez2022estimating}. Por otro lado, la transición de O a la EA es casi continua, con un número relativamente pequeño de genes expresados diferencialmente \cite{Gonzalez_2021}. Esto significa que existe una barrera muy pequeña, o incluso una ruta sin barrera, que conecta a O y a la EA. Esperamos una barrera de bajo \emph{fitness} que impida las transiciones directas de O a la EA, y un máximo de la EA pequeño, ya que este atractor se encuentra en la región de bajo \emph{fitness}, lejos de N. 

\begin{figure}[!htb]
	\centering
	\includegraphics[width=0.75\linewidth]{figures/Fig_1c.pdf}
	\caption{Diagrama de Wright que muestra un gráfico de contorno hipotético del \emph{fitness}. El máximo absoluto corresponde con el estado de GB. El atractor de la EA se representa como un ligero máximo local.}
	\label{fig:fig1c}
\end{figure}

Todos estos hechos se representan en la Fig. \ref{fig:fig1c}. El esquema se construye a partir de una suma de gaussianas centradas en los atractores, con desviaciones estándar proporcionales a los valores reales observados en la Fig. \ref{fig:fig1a} y con alturas que respetan cualitativamente la fuerza relativa de los atractores.

Resaltemos el significado de un diagrama de Wright en el tejido cerebral. En otros tejidos, la evolución somática se relaciona principalmente con la replicación de células madre. Sin embargo, en su estado normal, el cerebro es un tejido de replicación muy lenta \cite{spalding2005retrospective}. Los cambios en pequeñas regiones cerebrales, es decir, los desplazamientos en el diagrama, son básicamente daños acumulados, es decir, envejecimiento \cite{schumacher2021central}. Sin embargo, una vez que se produce la transición al estado GB, se produce un enorme aumento de la tasa de replicación de las células tumorales. Observemos, además, que los cambios relacionados con el envejecimiento son muy evidentes en la sustancia blanca \cite{guttmann1998white}.


\section{Limitaciones}\label{sec:limitations}

En este trabajo se utilizaron datos de expresión genética, en formato FPKM, de las referencias \cite{Brennan_2013, Miller_2017}. Estos fueron obtenidos usando diferentes plataformas. Nosotros tomamos aproximadamente 30 000 genes que están perfectamente identificados en ambas plataformas y realizamos un sencillo análisis de componentes principales \cite{Lever2017}, como se definió en la sección \ref{subsec:svd}. Para definir los valores de la expresión diferencial logarítmica y calcular la matriz de covarianza utilizada para el PCA, se utilizó como referencia común la media geométrica en el conjunto de muestras N.

Debido al uso de estos datos, proveniente de dos experimentos distintos, para realizar un solo calculo de PCA surgen problemas tanto técnicos como conceptuales. Por ejemplo, la referencia N corresponde precisamente al estado normal del cerebro, sino que son un conjunto de muestras patológicamente normales que fueron tomadas de individuos con GB. Además, dos de los pacientes tienen más de 70 años. Desde el punto de vista computacional, por otro lado, se podrían utilizar correcciones por lotes\cite{haghverdi2018batch, zhang2020combat}, que corrigen parcialmente los sesgos asociados a cada grupo de muestras, pero también pueden introducir problemas incontrolados.

En lugar de introducir procedimientos muy avanzados, preferimos extraer los datos directamente de las fuentes y utilizar la técnica de PCA más sencilla. No creemos que ninguna corrección altere sustancialmente el análisis cualitativo derivado del diagrama de tres atractores que se muestras en la Fig. \ref{fig:fig1a}.

La situación ideal podría ser repetir el experimento dentro de un único marco tecnológico, incluyendo datos de personas jóvenes sanas, que se utilizarían para establecer la referencia de los cálculos de la expresión genética diferencial, incluyendo datos de pacientes con GB y la EA, y datos de pacientes sanos en diferentes rangos de edad. Este es un experimento complejo, pero podría ser particularmente factible en un modelo de ratones \cite{hahn2023atlas}, por ejemplo. Consideramos nuestro diagrama de la Fig. \ref{fig:fig1a} como una aproximación cualitativa de este experimento ideal.
    % !TeX root = Tesis.tex
% !TeX encoding = UTF-8
% !TeX spellcheck = es_CU-SpanishCuba
\chapter{Resultados}
\label{cap3}
\onehalfspacing

\section{Principales resultados}

Sobre la base de nuestros diagramas, podemos formular las siguientes observaciones o afirmaciones, que son los principales resultados del trabajo.

\begin{enumerate}
	\item  \textbf{Existe una dirección en el espacio de expresión genética, que a grandes rasgos se puede identificar con el eje PC1, asociada al envejecimiento y a un aumento del riesgo de padecer GB y la EA.}
\end{enumerate}

De hecho, el desplazamiento en esta dirección implica escalar parcialmente las barreras de bajo \textit{fitness} que separa N de los estados GB y EA, y por lo tanto aumentar el riesgo tanto para GB como para la EA.

Vale la pena observar los principales genes involucrados en este proceso. Para ello, observamos el vector unitario a lo largo del eje PC1. Los genes se clasifican según su contribución al vector unitario.% \alert{El procedimiento es similar al algoritmo de Page Rank \cite{Duhan_2009}. Lo usamos en nuestro trabajo anterior \cite{Gonzalez_2023}.}

En la Tabla \ref{tab:apx3} de los apéndices se encuentra una lista con los 100 primeros genes del \textit{ranking}. En la Fig. \ref{fig:figpc1} se representan los valores de los 30 primeros genes en el vector PC1 ordenados por su valor absoluto. Las amplitudes positivas definen genes cuya expresión aumenta con el desplazamiento a lo largo de la dirección positiva de PC1, mientras que las amplitudes negativas se refieren a genes silenciados. Estos genes deberían desempeñar simultáneamente un papel crucial en el envejecimiento, el GB y la EA.

\begin{figure}[!htb]
	\centering
	\includegraphics[width=\linewidth]{figures/PC1}
	\caption{Genes con mayor peso a lo largo del vector PC1.}
	\label{fig:figpc1}
\end{figure}

Por supuesto, debido al valor puramente cualitativo de nuestro análisis, los genes, y especialmente el \textit{ranking}, deben considerarse con cautela. Sin embargo, cabe destacar que 20 de los genes silenciados están relacionados con los procesos principales de transmisión a través de sinapsis químicas. En la Tabla \ref{tab:apx4} del Apéndice \ref{apx:apx3} se enumeran los procesos principales del \textit{Reactome} (\href{https://reactome.org/}{https://reactome.org/}) asociadas a estos genes \cite{Gillespie_2021}. Este conjunto incluye 56 genes anotados.

La disminución de la función sináptica es una característica conocida del cerebro envejecido, según la revisión \cite{Ham_2020}. La segunda característica principal, según esta referencia, es un aumento de la función inmunitaria, que no es particularmente evidente en nuestro conjunto de genes. En cambio, observamos genes relacionados con la neurotoxicidad de las toxinas de clostridium \cite{Biazzo_2022}, con la disminución de la actividad mitocondrial \cite{Sun_2016}, micro ARN compartidos entre la EA y el GB \cite{Thomas_2020}, etc.

\begin{enumerate}
	\item[2.] \textbf{Hay una dirección en el espacio de expresión genética, que puede identificarse aproximadamente con el eje PC2, que muestra que la EA y GB son alternativas excluyentes.}
\end{enumerate}

De hecho, la EA y el GB aparecen en semiplanos opuestos. La evidencia clínica \cite{ou2012does, Driver_2012, Roe_2010, Musicco_2013} y los estudios de biología molecular \cite{Liu_2013, Lanni_2020} respaldan esta disyunción. En consecuencia, el eje PC2 involucra genes con desregulación inversa en la EA y el GB.

En la Tabla \ref{tab:apx5} de los apéndices, se listan los 100 genes principales definidos por el vector unitario a lo largo del eje PC2. En la Fig. \ref{fig:figpc2} se representa gráficamente la contribución a dicho vector de los 30 primeros genes ordenados por su valor absoluto. Los pesos positivos corresponden a genes cuya expresión aumenta en la transición de N a la EA. Por otro lado, las amplitudes negativas corresponden a genes cuya expresión aumenta en la transición de N a GB.

\begin{figure}[!htb]
	\centering
	\includegraphics[width=\linewidth]{figures/PC2}
	\caption{Genes con mayor peso a lo largo del vector PC2.}
	\label{fig:figpc2}
\end{figure}

Los procesos de \textit{Reactome} relacionadas con estos genes se muestran en la Tabla \ref{tab:apx6} del Apéndice \ref{apx:apx5}. Se relacionan principalmente con el control del ciclo celular, la replicación del ADN, la apoptosis, la modificación de la matriz extracelular, etc., es decir, con las características distintivas del cáncer \cite{Hanahan_2000, Hanahan_2011, Hanahan_2022}.

Anteriormente, mencionamos MMP9 como un ejemplo de genes que desempeñan funciones opuestas en el GB y la EA. El gen codificador de la proteína UBE2C es otro gen conocido con esta característica \cite{MA_2016, Jaladanki_2021}. Por otro lado, el gen BCYRN1, también entre los primeros 100 genes del \textit{ranking}, parece estar subexpresado en GB \cite{Mu2021} y sobreexpresado en la EA \cite{Zhang_2021}. La Fig. \ref{fig:fig1d} muestra gráficos de violín para la expresión diferencial de los genes MMP9 y BCYRN1 en muestras de N, GB y la EA.

\begin{figure}[!htb]
	\centering
	\includegraphics[width=0.75\linewidth]{figures/Fig_1d.pdf}
	\caption{Gráficos de violín para la expresión diferencial logarítmica de los genes MMP9 y BCYRN1 en los estados N, GB y EA.}
	\label{fig:fig1d}
\end{figure}

Noté también que en la Tabla \ref{tab:apx5} hay numerosos genes codificadores de proteínas ribosomales, de núcleo pequeño, de microARN y de otros tipos, regulados inversamente en ambos procesos. Solo 18 de este conjunto de genes están anotados en los procesos \textit{Reactome}. Este es un problema habitual en el análisis de procesos, donde las funciones biológicas de muchos genes no están anotadas.

\begin{enumerate}
	\item[3.] \textbf{Existe un corredor de envejecimiento, es decir un camino preferencial para el envejecimiento en el espacio de expresión genética.}
\end{enumerate}

En nuestros datos, existen muestras en la región N y muestras correspondientes a cerebros de edad normal, ubicadas en una región definida cerca del atractor de la EA. En otras palabras, el proceso de envejecimiento parece definir una trayectoria o corredor de decrecimiento continuo del \textit{fitness}, del cual los datos O muestran el último segmento. Sin embargo, faltan muestras en la región intermedia.

En lugar de incluir muestras adicionales en nuestra figura, lo cual introduciría efectos de lote adicionales, utilizamos resultados recientes en un modelo de ratón \cite{hahn2023atlas} que muestran indudablemente un corredor continuo para el envejecimiento. Presentamos en la Fig. \ref{fig:suppl1} del Apéndice \ref{apx:apx1} una representación gráfica de sus datos para el cuerpo calloso, una región rica en materia blanca. En el panel izquierdo, se grafican las dos primeras componentes principales para los centros de los subgrupos de muestras. Se consideran edades de ratón entre 3 y 28 meses; este último equivale aproximadamente a 80 años en una escala humana. Un corredor para el envejecimiento es evidente. El panel derecho, por otro lado, muestra distancias reales incluyendo todos los componentes. Por lo tanto, las proyecciones en el plano (PC1, PC2) son una representación fiel de la distribución real de puntos.

En nuestro esquema (Fig. \ref{fig:fig1b}), se delinea un corredor de envejecimiento. La Fig. \ref{fig:fig1c} sugiere que este corredor es una dirección con una mínima disminución del \textit{fitness}.

Una dirección o corredor preferencial para el envejecimiento es consistente con la hipótesis del envejecimiento programado \cite{Magalh_es_2012, Gems_2022}, es decir, la idea de que el envejecimiento está programado en nuestros genes.

\begin{enumerate}
	\item[4.] \textbf{El corredor de envejecimiento predeterminado podría estar relacionado con la presión de evitar el fuerte atractor GB.}
\end{enumerate}

Una pregunta muy interesante se refiere a la selección de la dirección preferida para el envejecimiento. Nuestro esquema simplificado (Fig. \ref{fig:fig1c}) ofrece una respuesta inesperada: en la materia blanca, esta dirección podría estar relacionada con la presión para evitar el atractor GB más fuerte.

De hecho, para cada pequeña porción de tejido, se puede modelar el envejecimiento como una especie de movimiento aleatorio que comienza en la región N. Obsérvese que en la referencia \cite{Herrero_2022} se utilizó un modelo de saltos aleatorios en el espacio de expresión genética para describir la evolución somática de diferentes tejidos hacia el cáncer. Primero, asumimos que la dirección de los saltos es aleatoria en el plano que se muestra en la Fig. \ref{fig:fig1c}.

Solo existen cuatro posibilidades para el destino de las trayectorias aleatorias en el plano que comienzan en la región N. Primero, la trayectoria permanece en la región N. Segundo, llega a una región de \textit{fitness} muy baja y las células mueren. Este es el destino de muchas células en el cerebro envejecido. Tercero, es capturada por el atractor GB. Y, finalmente, la cuarta posibilidad es la captura por el atractor de la EA.

Debido al elevado valor de \textit{fitness} del atractor GB, debería existir una probabilidad relativamente alta de que las trayectorias sean capturadas por el GB, lo que conduce a la iniciación de un tumor. Esto implica un enorme aumento del \textit{fitness}, la propagación del tumor en el cerebro y una esperanza de vida para el individuo de tan solo unos dos años tras la iniciación \cite{Poon_2020}. Esto podría afectar a los individuos en edad reproductiva. Por lo tanto, evitar el atractor GB podría ser objeto de la presión selectiva.

Como comprobación indirecta, podemos comparar las incidencias de GB y la EA. Estas deberían ser proporcionales a las tasas de captura de trayectorias aleatorias por los atractores de GB y EA. Como se mencionó, en un modelo donde la dirección de los saltos es aleatoria, la incidencia de GB debería ser mucho mayor que la de la EA. Sin embargo, la incidencia global del glioma es inferior a 10 por 100 000 personas \cite{Ohgaki2005}, en contraste con el 5 \% de la EA en personas de 65 a 74 años y el 13 \% en personas de 75 a 84 años \cite{alz2023}. La incidencia observada sugiere que se evita el movimiento hacia el centro de GB.

\begin{enumerate}
	\item[5.] \textbf{La aparición tardía de la EA podría ser el resultado de la captura por parte del atractor de la EA de microestados cerebrales envejecidos.}
\end{enumerate}

La imagen es, por lo tanto, la siguiente. El proceso de envejecimiento se relaciona inicialmente con un desplazamiento a lo largo del corredor de envejecimiento, con la correspondiente disminución del \textit{fitness}. En los últimos pasos, los estados O son capturados por el atractor débil de la EA.

Como se mencionó anteriormente, la afirmación sobre el corredor de envejecimiento está respaldada por el experimento en un modelo de ratón, mientras que la captura por parte del centro de la EA está sugerida por los cálculos de la referencia \cite{Gonzalez_2021}, particularmente por los resultados que se muestran en la Fig. \ref{fig:otoad}, que es un reconstrucción de la Fig. 4 de esa referencia.

En la Tabla \ref{tab:apx7} se muestran los 10 genes principales en la transición de O a la EA. Se trata de genes incluidos en la Tabla \ref{tab:apx3}, pero que varían en dirección opuesta, es decir, en la dirección negativa del eje PC1. Este hecho se representa en el diagrama esquemático de la Fig. \ref{fig:fig1b}.


\section{Perspectiva cuantitativa}

Debido al carácter cualitativo de nuestro estudio, el análisis de genes y procesos relevantes no se aborda adecuadamente en este artículo. Sin embargo, analicemos cualitativamente siete marcadores conocidos para la EA y el GB según nuestro esquema. Los gráficos de violín para estos genes se muestran en Fig. \ref{fig:violin}. La figura muestra que los genes MAPT (proteína tau \cite{Strang_2019}) y APP (beta amiloide \cite{TCW_2016}) están subexpresados tanto en el GB como en la EA y, por lo tanto, según nuestro esquema, son genes tipo PC1, principalmente relacionados con el envejecimiento. En cierto modo, esto concuerda con los hallazgos del estudio del Instituto Allen sobre los patrones de proteína tau y placas amiloides en el cerebro envejecido. Por supuesto, ciertas mutaciones de estos genes podrían conducir a un envejecimiento cerebral acelerado y a la aparición temprana de la EA. El gen APOE \cite{Raulin_2022}, por el contrario, está desregulado inversamente en el GB y la EA. Es un gen del tipo PC2.

\begin{figure}[!htb]
	\centering
	\includegraphics[width=\linewidth]{figures/suppl2}
	\caption{Gráficos de violín para marcadores conocidos en la EA y el GB.}
	\label{fig:violin}
\end{figure}

Por otro lado, el marcador IDH1 \cite{Cohen_2013} está sobreexpresado en la GB, pero es irrelevante en la EA. Los tres marcadores de GB, IDH2 \cite{Cohen_2013}, MKI67 \cite{Chen_2015} y ATRX \cite{Haase_2018}, son genes similares a PC2. Se han estudiado principalmente en relación con la GB, pero el hecho de que sean genes PC2 indica que podrían desempeñar un papel importante también en la EA.

Consideremos, como último ejemplo, la reciente demostración de la relevancia de TREM2 en la EA \cite{van_Lengerich_2023}. Se ha demostrado que la activación de TREM2 en la EA mejora el metabolismo microglial. Según nuestro análisis, TREM2 es un gen PC2, subexpresado en la EA y sobreexpresado en el GB. Por lo tanto, esperamos que la inhibición de TREM2 en células de glioma pueda tener un efecto importante en el GB. Este hecho, según nuestro sencillo esquema, se encuentra actualmente en estudio \cite{Sun_2023}.


%    \input{./cap4}
    % !TeX root = Tesis.tex
% !TeX encoding = UTF-8
% !TeX spellcheck = es_ES
\chapter*{Conclusiones}\label{conclutions}
\addcontentsline{toc}{chapter}{Conclusiones}
\onehalfspacing

\lipsum[12-13]


\chapter*{Recomendaciones}\label{recomendations}
\addcontentsline{toc}{chapter}{Recomendaciones}

\lipsum[14-15]

	
     %\input{./Pseudocodigo}
%     \input{./Ajuste}
     %Si hay más apéndices, solo los agregas
     


	\backmatter

	\singlespacing
	\bibliographystyle{ieeetr}  % or: plain,unsrt,alpha,abbrv,acm,apalike, ieeetr,..
%    \bibliographystyle{phd_bib}	% definir el estilo de la bibliografía .bst (phd_bib,mibib)
    \bibliography{bibl} 	% ejemplo. hacer uno genérico. Solo pone las que se citan.
%    \bibliography{test}
%    \nocite{*}

%	\singlespacing
%	\appendix
%%	\section*{Apéndices}
%	\addcontentsline{toc}{chapter}{Apéndices}
%	\input{./microarray-example}
%	\input{./tablas}

\end{document}
